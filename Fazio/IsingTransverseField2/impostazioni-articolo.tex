%*********************************************************************************
% impostazioni-articolo.tex
% di Lorenzo Pantieri (2013-2016)
% file che contiene le impostazioni dell'articolo
%*********************************************************************************


%*********************************************************************************
% Comandi personali
%*******************************************************
\newcommand{\myName}{Alessandro Candido}                            % autore
\newcommand{\myTitle}{1-D Ising Model in a Transverse Field}  % titolo
\date{\today}                                                       % oggi

\title{\normalfont\spacedallcaps{\myTitle}}
\author{\spacedlowsmallcaps{\myName}}



%*********************************************************************************
% Impostazioni di amsmath, amssymb, amsthm
%*********************************************************************************

% comandi per gli insiemi numerici (serve il pacchetto amssymb)
\newcommand{\numberset}{\mathbb} 
\newcommand{\N}{\numberset{N}} 
\newcommand{\R}{\numberset{R}} 

% un ambiente per i sistemi
\newenvironment{sistema}%
  {\left\lbrace\begin{array}{@{}l@{}}}%
  {\end{array}\right.}

% definizioni (serve il pacchetto amsthm)
\theoremstyle{definition} 
\newtheorem{definizione}{Definizione}

% teoremi, leggi e decreti (serve il pacchetto amsthm)
\theoremstyle{plain} 
\newtheorem{teorema}{Teorema}
\newtheorem{legge}{Legge}
\newtheorem{decreto}[legge]{Decreto}
\newtheorem{murphy}{Murphy}[section]


%*********************************************************************************
% Impostazioni di biblatex
%*********************************************************************************
\defbibheading{bibliography}{%
\manualmark
\phantomsection 
\addcontentsline{toc}{section}{\refname}
\section*{\bibname\markboth{\spacedlowsmallcaps{\refname}}
{\spacedlowsmallcaps{\refname}}}}



%*********************************************************************************
% Impostazioni di listings
%*********************************************************************************
\lstset{language=[LaTeX]Tex,%C++,
    keywordstyle=\color{RoyalBlue},%\bfseries,
    basicstyle=\small\ttfamily,
    %identifierstyle=\color{NavyBlue},
    commentstyle=\color{Green}\ttfamily,
    stringstyle=\rmfamily,
    numbers=none,%left,%
    numberstyle=\scriptsize,%\tiny
    stepnumber=5,
    numbersep=8pt,
    showstringspaces=false,
    breaklines=true,
    frameround=ftff,
    frame=single
} 



%*********************************************************************************
% Impostazioni di hyperref (decommenta le seguenti righe se non carichi arsclassica)
%*********************************************************************************
%\hypersetup{%
%    hyperfootnotes=false,pdfpagelabels,
%    %draft,	% = elimina tutti i link (utile per stampe in bianco e nero)
%    colorlinks=true, linktocpage=true, pdfstartpage=1, pdfstartview=FitV,%
%    % decommenta la riga seguente per avere link in nero (per esempio per la stampa in bianco e nero)
%    %colorlinks=false, linktocpage=false, pdfborder={0 0 0}, pdfstartpage=1, pdfstartview=FitV,% 
%    breaklinks=true, pdfpagemode=UseNone, pageanchor=true, pdfpagemode=UseOutlines,%
%    plainpages=false, bookmarksnumbered, bookmarksopen=true, bookmarksopenlevel=1,%
%    hypertexnames=true, pdfhighlight=/O,%nesting=true,%frenchlinks,%
%    urlcolor=webbrown, linkcolor=RoyalBlue, citecolor=webgreen, %pagecolor=RoyalBlue,%
%    %urlcolor=Black, linkcolor=Black, citecolor=Black, %pagecolor=Black,%
%    pdftitle={\myTitle},%
%    pdfauthor={\textcopyright\ \myName},%
%    pdfsubject={},%
%    pdfkeywords={},%
%    pdfcreator={pdfLaTeX},%
%    pdfproducer={LaTeX with hyperref and ClassicThesis}%
%}



%*********************************************************************************
% Impostazioni di graphicx
%*********************************************************************************
\graphicspath{{Immagini/}} % cartella dove sono riposte le immagini



%*********************************************************************************
% Altro
%*********************************************************************************

% molto utile (Ale)
\usepackage[sort, noabbrev, capitalise]{cleveref}

% [...] ;-)
\newcommand{\omissis}{[\dots\negthinspace]}

% eccezioni all'algoritmo di sillabazione
\hyphenation{Fortran ma-cro-istru-zio-ne nitro-idrossil-amminico}

\areaset[current]{370pt}{750pt}
\setlength{\marginparwidth}{7em}
\setlength{\marginparsep}{2em}%