\documentclass[a4paper,10pt]{article}
\usepackage[utf8]{inputenc}
\usepackage[T1]{fontenc}	
\usepackage[italian]{babel}

\usepackage{amsmath}
\usepackage{amsfonts}
\usepackage{amssymb}
\usepackage{graphicx}

\usepackage[left=2cm,right=2cm,top=2cm,bottom=2cm]{geometry}
\geometry{a4paper}

\usepackage{booktabs}
\usepackage{verbatim}
\usepackage{subfig}
\usepackage[dvipsnames]{xcolor}  %colori
\usepackage[colorlinks=true, linkcolor=black, urlcolor=blue, citecolor=darkgray, filecolor=darkgray]{hyperref}   %per gli hyperlink

\usepackage{braket}

\usepackage[cdot, thickqspace, squaren]{SIunits}
\usepackage{float}

% macro
\def\code#1{\texttt{#1}}

\title{Tesi triennale\\ Appunti in corso d'opera}
\author{Alessandro Candido}

\begin{document}

\maketitle

\section{Coda di lavoro}
Scrivo qui di seguito la lista delle cose pratiche da fare in ordine, almeno quelle che sono ben note, così da avere il lavoro pronto quando ho le energie per farlo:

\begin{itemize}
	\item Polkovnikov, finire di leggere;
	\item Articoli sperimentali 2$^a$ mandata;
	\item Articoli teorici 2$^a$ mandata;
	\item Teoria di Floquet, il capitolo originale della 1$^a$ mandata.
\end{itemize}

La lista è in divenire, però mi propongo di inserire le cose in coda per lo più.

\section{Argomenti}

Anche qui la lista è ordinata, potrei divertirmi a fare un bel grafico tipo diagramma di Venn, magari c'è un pacchetto di \LaTeX~ che lo fa, ma per il momento mi limiterò alla lista pseudoordinata, anche perché c'è un certo ordinamento indotto da una specie di propedeuticità.

\begin{itemize}
	\item GGE (generalised Gibbs ensemble);
	\item Ising model (Lieb, McCoy, Baruch, ...);
	\begin{itemize}
		\item Tilted Ising model;
		\item Numerical studies;
	\end{itemize}
	\item Floquet Theory;
	\begin{itemize}
		\item Periodically driven systems;
		\item Starting a perturbation;
		\item Kicked systems are grouped in quantum chaos;
	\end{itemize}
	\item Quantum chaos: quantum rotator (cylinder phase space, toroidal phase space);
	\begin{itemize}
		\item Kicked quantum rotator;
		\item Classical chaos
		\item KAM theory
	\end{itemize}
\end{itemize}

\section{Out of equilibrium dynamics}
Nella review di Polkovnikov si trova un punto della situazione rispetto allo spettro delle possibilità di dinamica fuori equilibrio. Rossini ha detto che questa review in particolare è considerata un punto di riferimento per il settore, e io la considererò tale.

Apparentemente ci sono due macroclassi studiate:
\begin{itemize}
	\item \textbf{Dinamica quasi adiabatica:} Ci si concentra su un  modello di evoluzione adiabatica, e si va a studiare le deviazioni da questo modello, tipo teoria di Landau-Zener-Majorana; ci si aspetta di osservare deviazioni nell'intorno di punti critici e transizioni di fase in generale;
	\item \textbf{Dinamica "veloce":} Ne so ancora poco, appena scopro qualcosa di più lo scrivo, comunque è quello che devo fare, \textbf{keywords:} ETH, GGE, pseudotermalizzazione.	
\end{itemize}

\paragraph{Dinamica quasi adiabatica}
Due argomenti di scaling:
\begin{itemize}
	\item \textbf{classico:} dal confronto del tempo residuo di avvicinamento;
	\item \textbf{quantum:} dal confronto dell'energy gap (fondamentale-primo eccitato).
\end{itemize}

Tante altre cose che posso raccattare rileggendo la parte corrispondente di Polkovnikov.
Comunque si concentra molto sulla generazione di eccitazioni locali, e la densità di queste (per cui anche la dimensione tipica dei domini coerenti e le varie lunghezze di coerenza).

Ci sono anche un po' di cose riguardo scambi di calore e varie definizioni di entropia: entropia di Von Neumann, entropia di Polkovnikov.

\paragraph{Dinamica "veloce"} C'è da leggere Polkovnikov, ha poco senso scrivere qualcosa perché descriverei a spanne quello che andrò a leggere da lì.

\section{Generalised Gibbs Ensemble}
Si vorrebbe introdurre un ensemble alla Boltzmann su cui fare le medie, ma ovviamente questo implicherebbe una situazione di equilibrio, a questo punto si diagonalizza l'Hamiltoniana con una base di quasiparticelle (numeri d'occupazione):

\begin{equation}
\mathcal{H} = \sum_q \varepsilon_q \hat{n}_q
\end{equation}

\noindent dove gli $\hat{n}_q$ sono gli operatori numero.
A questo punto uno incrementa il numero id moltiplicatori di Lagrange e definisce più "pseudotemperature" $\beta_q$, una per ogni possibile stato di quasiparticella.

\begin{equation}
\rho(\vec{n}) = \frac{1}{Z} e^{- \sum_q \beta_q \hat{n}_q}
\end{equation}

\noindent dove le nuove temperature $\beta_q$ si inglobano anche le scale di energia $\varepsilon_q$ e diventano variabili adimensionali.\footnote{where $\hat{n}_q$	is  a  set  of  nontrivial  conserved  quantities  that  exists because  the  system  is integrable, \href{https://arxiv.org/pdf/1604.03990.pdf}{Rigol article}}

\section{Ising model}

\subsection{Transverse Ising model} Si risolve esattamente, è stato fatto per la prima volta negli anni $'70$ da McCoy (o forse per lo più era già stato fatto negli anni $'60$ da Lieb). Risolvere significa che si è ricavato tutto lo spettro.

L'Hamiltoniana è del tipo:

\begin{equation}
\mathcal{H} = -~ J \sum_j \sigma_j^{(x)} \sigma_{j+1}^{(x)} ~-~ h \sum_k \sigma_k^{(z)}
\end{equation}

Su questa Hamiltoniana si ha una competizione tra il primo termine, che tende ad allineare gli spin tutti nello stesso stato lungo $\ket{\uparrow_x}$ oppure $\ket{\downarrow_x}$, mentre il secondo termine tende a portare tutti gli stati nello stesso, ad esempio $\ket{\uparrow_z}$ (da notare che la simmetria è rotta a causa del campo $h$), e quindi entra in competizione col primo poiché cerca di mantenere una certa sovrapposizione (simmetrica o antisimmetrica) degli stati di spin lungo $x$.

Si ha una transizione di fase a ground state, per cui un QCP a temperatura nulla, che si ottiene per valori delle costanti di accoppiamento $J/h = 1$.

\paragraph{Situazione banale} Se anziché prendere le matrici $\sigma$ lungo due direzioni diverse scelgo sempre le matrici lungo sempre la stessa direzione banalizzo il problema e torno a variabili classiche: posso diagonalizzare l'Hamiltoniana in modo banale direttamente con le variabili di spin e ritrovo il modello di Ising classico unidimensionale che non presenta più niente di interessante.

\paragraph{Idea, non fisica} E se provassi ad accoppiare gli spin in modo bislacco? Tipo:
\begin{equation}
\mathcal{H} = -~ J \sum_j \sigma_j^{(x)} \sigma_{j+1}^{(z)}
\end{equation}

\paragraph{Idea, più fisica} E per un modello in cui gli spin sono $3$D, per cui ho l'accoppiamento in tutte le direzioni? (Interazione spin-spin a primi vicini)


\begin{equation}
\mathcal{H} = -~ J \sum_j \vec{\sigma}_j \cdot \vec{\sigma}_{j+1}
\end{equation}

\noindent A questo punto il campo posso metterlo nella direzione che mi pare, tanto sono tutte equivalenti.

E se ancora avessi un interazione spin-spin diversa per le varie direzioni?

\begin{equation}
\mathcal{H} = -~ J \sum_j \gamma_{\alpha, \beta} ~ \sigma_j^{(\alpha)} \cdot \sigma_{j+1}^{(\beta)}
\end{equation}

\noindent E qui il campo boh, ma anche tutto il resto non ci capisco nulla


\subsection{Tilted Ising model} Se inizio a provare a inserire nell'Hamiltoniana altri termini (banalmente campo non lungo le direzioni coordinate, parallele o ortogonali, ma storto a casaccio), il problema inizia a diventare parecchio difficile, e non è più risolubile esattamente:

\begin{equation}
\mathcal{H} = -~ J \sum_j \sigma_j^{(x)} \sigma_{j+1}^{(x)} ~-~ h^{(z)} \sum_k \sigma_k^{(z)} ~-~ h^{(x)} \sum_l \sigma_l^{(x)}
\end{equation}

\section{Floquet Theory}
Boh, apparentemente è bella e basta, ma ci devo ancora arrivare.

\section{Quantum chaos}

\subsection{Quantum kicked rotator} Si prende un rotatore, per semplicità, cioè un sistema $1$D con una variabili di angolo $\in [0,2\pi]$ e il suo impulso coniugato (che sarà un momento angolare unidimensionale necessariamente quantizzato).

Lo si concentra in pacchetti gaussiani di minima indeterminazione (nella variabile angolare) e lo si lascia evolere sotto l'azione di un Hamiltoniana che, a intervalli di tempo costanti, gli da un impulso (kick) dipendente dall'angolo, che può essere espresso come un operatore che agisca sull'angolo (costante nel tempo) moltiplicato per tante $\delta$ di Dirac nei tempi, equispaziate.\footnote{Penso si possano considerare anche operatori diversi nei tempi, ma il problema si complica senza aggiungere niente di interessante, a meno che non si preveda un certo tipo di evoluzione (una dinamica) per questi kick}.

Per certi valori dei parametri di accoppiamento il sistema è caotico.

\paragraph{Irreversibilità numerica} Mentre in meccanica quantistica l'evoluzione temporale è più regolare e controllabile, se mi metto a far evolvere un sistema caotico classico, e poi inverto l'evoluzione temporale, potrei non essere neanche lontanamente in grado, a livello numerico, di tornare al punto di partenza seguendo la dinamica.

I sistemi classici presentano più instabilità di quelli quantistici $^{[\textit{senza fonte}]}$.

\section{Experimental results}

\subsection{Quantum Newton's cradle} Generano due condensati, confinati in un potenziale "a sigaro" (parabolico e stretto in due direzioni ortogonali e piatto al centro, e magari nuovamente parabolico ai lati, e allungato nella terza), indistinguibili (penso) e localizzati ai margini del lato lungo, con impulsi uguali e opposti (e convergenti, ma dovrebbe essere uguale a meno di una fase temporale).

Li lasciano andare e vedono che continuano a oscillare anziché termalizzare.

\'E importante perché è stato il punto di partenza per altri lavori sperimentali e anche per tutto il filone teorico.

\section{Glossario}

\begin{description}
	\item[QCP:] punto critico quantistico, corrisponde a una transizione di fase a temperatura nulla (quindi a ground state) e può essere del secondo ordine, ma può anche essere associata a una discontinuità nell'entropia (come faccio ad avere un salto, non sono sempre a entropia nulla?);
	\item[quench:] è il modo più facile per portare un sistema fuori equilibrio, cioè una brusca variazione di qualche parametro dell'Hamiltoniana;
	\item[density of excitations:]
	\item[correlation length:]
	\item[gapped/gapless systems:] è associata al gap al limite termodinamico fra il fondamentale e il primo/i primi eccitati, i due tipi di sistemi presentano caratteristiche sostanzialmente differenti;
\end{description}

\end{document}