\documentclass[a4paper,10pt]{article}
\usepackage[utf8]{inputenc}
\usepackage[T1]{fontenc}	
\usepackage[italian]{babel}

\usepackage{amsmath,amsfonts,amssymb,amsthm}
\usepackage{mathtools,bbold,physics}

\usepackage{graphicx}
\usepackage[dvipsnames]{xcolor}  %colori

\usepackage[left=2cm,right=2cm,top=2cm,bottom=2cm]{geometry}
\geometry{a4paper}

\usepackage{verbatim}
\usepackage{lipsum}

\usepackage{booktabs}
\usepackage{subfig}
\usepackage{float}
\usepackage{multicol}

\usepackage[colorlinks=true, linkcolor=MidnightBlue, urlcolor=blue, citecolor=Emerald, filecolor=RoyalBlue]{hyperref}   %per gli hyperlink
\usepackage[italian, sort, noabbrev, capitalise]{cleveref}
\usepackage[bottom]{footmisc}

\usepackage[cdot, thickqspace, squaren]{SIunits}

% macro
\def\code#1{\texttt{#1}}

\title{Experimental Evidences}
\author{Alessandro Candido}

\begin{document}

\maketitle
\tableofcontents

\section*{}

La nota principale sulla tesi è \href{./Appunti.pdf}{Appunti.pdf}, guarda quella per conoscere gli sviluppi globali.

\section{Quantum phase transition: \\ from a superfluid to a Mott insulator \\ \textit{M.Greiner et al.}}

\paragraph{Sintesi:} si realizza un modello di Bose-Hubbard con una trappola magnetica ed un reticolo ottico 3D e si osserva la transizione quantistica superfluido-isolante di Mott variando l'intensità del reticolo (2002).

\section{Quantum Newton's cradle \\ \textit{T. Kinoshita et al.}}

\paragraph{Sintesi:}

\section{Light-cone-like spreading of correlations \\ \textit{M. Cheneau et al.}}

\paragraph{Sintesi:}

\section{Relaxation and Prethermalization \\ in an Isolated Quantum System \\ \textit{M. Gring et al.}}

\paragraph{Sintesi:}

\section{Many-body localization of interacting fermions \\ in a quasirandom optical lattice\\ \textit{M. Schreiber et al}}

\paragraph{Sintesi:}

\end{document}