% Pantieri lo chiama colophon, va bene per scrivere qualcosa in seconda pagina.
\phantomsection
\thispagestyle{empty}

\hfill

\vfill

\noindent Alessandro Candido: \textit{Appunti - Struttura della materia,}
%tipo di opera,
\textcopyleft\ \DTMMonthname{\the\month} \the\year
\newline

\lipsum[2]

\pdfbookmark{\contentsname}{tableofcontents}
\setcounter{tocdepth}{2}
\tableofcontents
\markboth{\scshape{\contentsname}}{\scshape{\contentsname}}

\clearpage
\phantomsection
\thispagestyle{empty}
\pdfbookmark{Dedica}{Dedica}

\vspace*{3cm}

\begin{quote}
	Felix qui potuit rerum cognoscere causas, [...] \\
	Fortunatus et ille deos qui novit agrestis, \\ \medskip
	--- P. Vergilius Maro, \textit{Georgicon}
\end{quote}

\medskip

\begin{center}
	Dedica generica, o altra frase se gira diversamente.
	Usare la dedica solo per riempire una pagina di sinistra, cioè se l'indice occupa un numero dispari di pagine, altrimenti \textbf{niente dedica}.
\end{center}

\pdfbookmark{Introduzione}{introduzione}

\chapter*{Introduzione}

Descrizione del contenuto dell'opera e motivazioni/altre chiacchiere varie.
\newline

\lipsum[1]
