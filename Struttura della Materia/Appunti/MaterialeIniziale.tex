% Pantieri lo chiama colophon, va bene per scrivere qualcosa in seconda pagina.
\phantomsection
\thispagestyle{empty}

\hfill

\vfill

\noindent Alessandro Candido: \textit{Appunti - Struttura della materia,}
%tipo di opera,
\textcopyleft\ \DTMMonthname{\the\month} \the\year
\newline

\pdfbookmark{\contentsname}{tableofcontents}
\setcounter{tocdepth}{2}
\tableofcontents
\markboth{\scshape{\contentsname}}{\scshape{\contentsname}}

\clearpage
\phantomsection
\thispagestyle{empty}
\pdfbookmark{Dedica}{Dedica}

\vspace*{3cm}

\begin{quote}
	Felix qui potuit rerum cognoscere causas, [...] \\
	Fortunatus et ille deos qui novit agrestis, \\ \medskip
	--- P. Vergilius Maro, \textit{Georgicon}
\end{quote}

\medskip

\begin{center}
	Se mi viene in mente qualche idea migliore cambio dedica, ma proprio non mi andava di dedicare delle dispense a qualche mio congiunto, per cui credo di fare la cosa più sensata \textbf{dedicando} queste dispense agli \textbf{studenti di fisica}, gentaccia per lo più... ma in fondo la maggior parte mi stanno simpatici.
\end{center}

\pdfbookmark{Introduzione}{introduzione}

\chapter*{Introduzione}

Queste dispense sono state scritte sulla base del corso di \textit{Struttura della Materia} tenuto dal prof. \textit{A.Tredicucci} nell'anno accademico \textit{2016-2017}.
\newline

A mio parere il corso non ha realmente bisogno di dispense: ognuno può avere il suo parere sulla chiarezza delle lezioni, ma non c'è dubbio che esse non siano ben documentate. Dall'anno in corso il professore ha regolarmente riportato i libri consultati di volta in volta nel \href{http://unimap.unipi.it/registri/dettregistriNEW.php?re=181626::::&ri=12126}{registro delle lezioni}\footnote{Per chi legge a schermo il link è cliccabile, per la stampa non ha senso che io scriva qui l'URL, per cui cercate "Tredicucci registro delle lezioni" su google.} (un utile strumento che si consiglia di consultare), per cui il materiale è disponibile e in abbondanza.
\newline

Ho scritto quindi queste dispense principalmente a mio uso e consumo, per cui insieme a parti fedelmente riportate (in modo possibilmente sintetico, ma spesso privilegiando la completezza) troverete anche pezzi interamente scritti da me: non sto affermando di vantare un contributo diretto agli argomenti trattati, ma piuttosto avvertendo il lettore su ciò che troverà. Le riflessioni presenti sono frutto in parte delle lezioni e in parte delle idee che mi sono venute cercando di chiarire alcuni punti più o meno oscuri (infatti molte riguardano principi e questioni sui fondamenti). Esse sono chiare per me, ma non è detto che lo siano per tutti, per cui siate critici e cercate di farvi la vostra opinione.

\begin{flushright}
	Buon lavoro,\\
	Alessandro
\end{flushright}
